%\documentclass[smallextended]{svjour3}
\documentclass[11pt]{report}
%\pagestyle{empty}

%\usepackage[latin1]{inputenc}
\usepackage{amssymb}
\usepackage{enumerate}
\usepackage{graphicx}
\usepackage[many,breakable]{tcolorbox}
\usepackage{natbib}

%\usepackage{xr}
%\externaldocument{NegriEA_AssemblyUCD_R1}

\usepackage[utf8]{inputenc}
\usepackage[T1]{fontenc}

\newtcolorbox{mybox}[1]{
colback = white!5!white,
colframe = white!75!black,
fonttitle=\bfseries,
title=#1,breakable
}

\usepackage{graphicx}
\usepackage{subfigure}

%\newtcolorbox{mybox}[1]{
%colback = gray!5!white,
%colframe = gray!75!black,
%fonttitle=\bfseries,
%title=#1
%}

%%%% determina o tamanho da folha
\usepackage[pdftex]{geometry}
\geometry{a4paper,left=2.8cm,right=2.8cm,top=1.5cm,bottom=1.5cm,twoside} % tamanho da pagina

%%% --------------------------------


%-----------------------------------------------------------------------------------
\begin{document}


\begin{center}
\large{\textbf{Response Letter}}

\vglue 0.3cm

\huge{Wavelet Spatio-Temporal Change Detection on multi-temporal SAR images}
\end{center}

%\author
\begin{center}
\textbf{Rodney~V.~Fonseca, Rog\'{e}rio~G.~Negri, Alu\'{i}sio~Pinheiro and Abdourrahmane~Atto}
\end{center}

\date{\today}


\vspace{2cm}
\noindent Dear Associate Editor,
\bigskip

\textcolor{black}{We would like to thank the reviewers for the valuable comments and the time spent on checking our manuscript. 
We have addressed their comments towards further improving the paper contents and presentation. 
Below we detail the changes and updates incorporated into the manuscript in this round of review.}


\medskip
\noindent Yours sincerely,

\begin{flushright}
\noindent The authors.
\end{flushright}
%-----------------------------------------------------------------------------------




%%%%%%%%%%%%%%%%%%%
\noindent---------------------------------------------------------------------------------------
\section*{Editor's decision}

%-------------------------------------
\textit{
Dear Dr. Fonseca:\\
I am sorry to inform you that your manuscript:\\
Wavelet Spatio-Temporal Change Detection on multi-temporal POLSAR images (TGRS-2021-01131)\\
has been carefully reviewed by the Editorial Board of the IEEE Transactions on Geoscience and Remote Sensing
(TGRS) and has not been recommended for publication at this time. It is felt that, while there may be sufficient original
content in the manuscript to warrant publication, it still requires a more substantial amount of work than is consistent with
a ‘Major Revision’ decision. For this reason, we encourage you to consider revising the manuscript in response to the
comments given below and submitting it as a new submission to TGRS. If you do decide to resubmit the paper, please
include a point-by-point response to the comments of the reviewers along with the new paper in order to expedite its
review. Use the "Submit a Resubmission" link in your author center when you submit the new manuscript.\\
Your resubmission deadline is 20-Jan-2022.\\
Below are summary comments from the TGRS Editorial Board for your information.\\
Thank you for submitting your manuscript to TGRS.\\
Sincerely,\\
Dr. Simon Yueh\\
Editor, IEEE Trans. on Geoscience and Remote Sensing
}

\medskip

\textbf{R:} .
%-------------------------------------

%%%%%%%%%%%%%%%%%%%
\noindent---------------------------------------------------------------------------------------
\section*{Associate Editor}

%-------------------------------------
\textit{Dear Authors,\\
I regret to inform you that I cannot recommend publication of your paper in the IEEE Transactions on Geoscience and
Remote Sensing journal. This rejection is based upon the enclosed comments of the reviewers.
In case you decide to resubmit the paper to this journal, I suggest that you take into account all the reviewers’ comments
to improve your manuscript.
Thank you for considering TGARS for the publication of your manuscript. We welcome the future submission of your
work to the journal.\\
Sincerely,}

\medskip

\textbf{R:} .
%-------------------------------------

\medskip


\newpage

\vspace{0.25cm}

%%%%%%%%%%%%%%%%%%%
\noindent---------------------------------------------------------------------------------------
\section*{Reviewer \#1}

%-------------------------------------
\textit{This paper presents a wavelet decomposition based change detection method for POLSAR images. The method seems original but the paper is poorly written. The detailed comments are listed as follows.}

\medskip

\textbf{R:} .
%-------------------------------------

\medskip


\medskip
%-------------------------------------
\begin{mybox}{Comment 1}
\textit{This paper lacks a thorough review of POLSAR image change detection methods. What are the challenges of
POLSAR image change detection? What are the advantages and defects of existing methods? What’s the superiority of
the proposed method over existing ones? What are the contributions of this paper to remote sensing community? There
are many works about change detection in POLSAR images. This paper only addresses the method without necessary
comparison.}

\medskip

\textbf{R:} .


\medskip




\end{mybox}
%-------------------------------------

\vspace{0.3cm}





\medskip
%-------------------------------------
\begin{mybox}{Comment 2}
\textit{The authors claim that “it is sparse, very fast and scalable. Finally, it is easily adapted to be updatable”. However, the
change detection in POLSAR images has been significantly developed and there are many excellent methods. Many of
them are also fast and scalable. Without any necessary comparison, how to demonstrate that “it is very fast and scalable”? What’s the novelty of the proposed method?}

\medskip

\textbf{R:} .

\medskip

\end{mybox}
%-------------------------------------

\vspace{0.3cm}


\medskip
%-------------------------------------
\begin{mybox}{Comment 3}
\textit{What’s the relationship between this paper and [i]? [i] also uses wavelet transformation for change detection.
[i] Fonseca, R. , Ludwig, G. , Montoril, M. , Pinheiro, A. . (2020). Nonparametric methods for detecting change in Multitemporal SAR-PolSAR Satellite Data. arXiv: 2001.05764.}

\medskip

\textbf{R:} .

\medskip


\end{mybox}
%-------------------------------------

\vspace{0.3cm}


\medskip
%-------------------------------------
\begin{mybox}{Comment 4}
\textit{Methodology describes the proposed method in detail, but what’s new in this paper? All the operators are existing or
intuitive.}

\medskip

\textbf{R:} .

\medskip


\end{mybox}
%-------------------------------------

\vspace{0.3cm}

\medskip
%-------------------------------------
\begin{mybox}{Comment 5}
\textit{The method is not clearly described. For example, what’s log-image? What’s the epsilon in Eq. (1). Eq. (1) seems
problematic. It denotes wavelet decomposition?}


\medskip
\textbf{R:} .

\medskip

\end{mybox}
%-------------------------------------

\vspace{0.3cm}

\medskip
%-------------------------------------
\begin{mybox}{Comment 6}
\textit{Without comparison with recent change detection methods in experiments, how to demonstrate the advantages of the
proposed method? For example, if there is a method that is faster and performs better than existing ones, why I have to
use the proposed one?}


\medskip
\textbf{R:} .

\medskip

\end{mybox}
%-------------------------------------

\vspace{0.3cm}

\medskip
%-------------------------------------
\begin{mybox}{Comment 7}
\textit{The authors claims that “Its performance is similar to a deep learning feature extraction method”. However, the deep
learning feature extraction method is proposed for image denoising and it is not specially designed for POLSAR images.
It is obviously unfair to compare the proposed method with it. Why not compare the proposed method with recent
POLSAR image change detection methods? The proposed one cannot outperform them?}


\medskip
\textbf{R:} .

\medskip

\end{mybox}
%-------------------------------------

\vspace{0.3cm}

\medskip
%-------------------------------------
\begin{mybox}{Comment 8}
\textit{For real data, there are only self comparisons without any comparison with existing excellent methods. Therefore, the
conclusions may be not correct.}


\medskip
\textbf{R:} .

\medskip

\end{mybox}
%-------------------------------------



\newpage



\vspace{0.25cm}

%%%%%%%%%%%%%%%%%%%
\section*{Reviewer \#2}


%-------------------------------------
\begin{mybox}{General Comment}

\textit{Fonseca et al. present a wavelet-based approach to detect spatiotemporal change points on multi-temporal Polarimetric
SAR. The approach is potentially interesting and might be of use for a broader community. However, it suffered from
several shortcomings, which prevents me from recommending the publication of the manuscript in its current form. In the
following o elaborate my concerns, and I hope authors find them useful.}

\medskip

\textbf{R:}  .
\end{mybox}
%-------------------------------------

\medskip
%-------------------------------------
\begin{mybox}{Comment 1}
\textit{the method section is unclear and partly incomplete. There are equations whose terms are not defined (e.g., Eq 1 and
that between 7 and 8 without a number). Furthermore, the procedure for Wavelet multiresolution analysis is not
described. Although, provided in the cited literature, it is crucial to carefully layout the procedure used to create the input
to the change detection algorithm. Also, statements like "J-th approximation level" are unclear. I suspect the author
meant to say "J-th decomposition level."}


\medskip
\textbf{R:} .
\end{mybox}
%-------------------------------------

\medskip

%-------------------------------------

\begin{mybox}{Comment 2}
\textit{synthetic test is confusing. Figure 1 shows a sequence of 4 synthetic images with changes appearing consecutively.
The expectation from the presented method and promises made in the abstract and introduction is that 4 corresponding
change maps can be obtained characterizing the changes made in the synthetic datasets. However, Figure 2 shows a
total change and some other maps, which is unclear how they were obtained or what is the relevance to the presented
method and promises of the paper.}

\medskip

\textbf{R:} .
\end{mybox}
%-------------------------------------


%-------------------------------------

\vspace{0.3cm}


\medskip
%-------------------------------------

\begin{mybox}{Comment 3}
\textit{the real data set is not introduced carefully, and it is not clear what datasets from what sensor at what resolution and
quality are used. How the real data is proceeded and prepared for this analysis, etc. In Figures 7-9 most of the panels
(e.g., c-j) show many overlapping curves that color-coded without providing a colorbar. These figures carry very little
useful information and I encourage authors to rethink ways to present their results. I also suggest authors to provide
examples of images and show the wavelet decomposition results. Also, show the average approximate coefficient
matrix. Last but not least, quantify a change map and provide a validation test such as an independent optical image
taken from the same area...}

\medskip

\textbf{R:} .
\end{mybox}
%-------------------------------------


%-------------------------------------




\vspace{0.3cm}


\bibliographystyle{apalike}
\bibliography{../JARS_DA_review/refsartigo}


\end{document}



