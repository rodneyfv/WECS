%\documentclass[smallextended]{svjour3}
\documentclass[11pt]{report}
%\pagestyle{empty}

%\usepackage[latin1]{inputenc}
\usepackage{amssymb}
\usepackage{enumerate}
\usepackage{graphicx}
\usepackage[many,breakable]{tcolorbox}
\usepackage{natbib}

%\usepackage{xr}
%\externaldocument{NegriEA_AssemblyUCD_R1}

\usepackage[utf8]{inputenc}
\usepackage[T1]{fontenc}

\newtcolorbox{mybox}[1]{
colback = white!5!white,
colframe = white!75!black,
fonttitle=\bfseries,
title=#1,breakable
}

\usepackage{graphicx}
\usepackage{subfigure}

%\newtcolorbox{mybox}[1]{
%colback = gray!5!white,
%colframe = gray!75!black,
%fonttitle=\bfseries,
%title=#1
%}

%%%% determina o tamanho da folha
\usepackage[pdftex]{geometry}
\geometry{a4paper,left=2.8cm,right=2.8cm,top=1.5cm,bottom=1.5cm,twoside} % tamanho da pagina

%%% --------------------------------


%-----------------------------------------------------------------------------------
\begin{document}


\begin{center}
\large{\textbf{Response Letter}}

\vglue 0.3cm

\huge{Wavelet Spatio-Temporal Change Detection on multi-temporal SAR images}
\end{center}

%\author
\begin{center}
\textbf{Rodney~V.~Fonseca, Rog\'{e}rio~G.~Negri, Alu\'{i}sio~Pinheiro and Abdourrahmane~Atto}
\end{center}

\date{\today}


\vspace{2cm}
\noindent Dear Editors,
\bigskip

\textcolor{black}{We would like to thank you and the reviewers for the valuable comments and the time spent on checking our manuscript. 
We have addressed their comments towards further improving the paper contents and presentation. 
Below we detail the changes and updates incorporated into the manuscript in this new submission.}


\medskip
\noindent Yours sincerely,

\begin{flushright}
\noindent The authors.
\end{flushright}
%-----------------------------------------------------------------------------------




%%%%%%%%%%%%%%%%%%%
\noindent---------------------------------------------------------------------------------------
\section*{Editor's decision}

%-------------------------------------
\textit{
Dear Dr. Fonseca:\\
I am sorry to inform you that your manuscript:\\
Wavelet Spatio-Temporal Change Detection on multi-temporal POLSAR images (TGRS-2021-01131)\\
has been carefully reviewed by the Editorial Board of the IEEE Transactions on Geoscience and Remote Sensing
(TGRS) and has not been recommended for publication at this time. It is felt that, while there may be sufficient original
content in the manuscript to warrant publication, it still requires a more substantial amount of work than is consistent with
a ‘Major Revision’ decision. For this reason, we encourage you to consider revising the manuscript in response to the
comments given below and submitting it as a new submission to TGRS. If you do decide to resubmit the paper, please
include a point-by-point response to the comments of the reviewers along with the new paper in order to expedite its
review. Use the "Submit a Resubmission" link in your author center when you submit the new manuscript.\\
Your resubmission deadline is 20-Jan-2022.\\
Below are summary comments from the TGRS Editorial Board for your information.\\
Thank you for submitting your manuscript to TGRS.\\
Sincerely,\\
Dr. Simon Yueh\\
Editor, IEEE Trans. on Geoscience and Remote Sensing
}

\medskip

\textbf{R:} We would like to thank the editor for his comments and for giving us the opportunity to improve the manuscript and submit it again to TGRS. Another coauthor has joined our group and we have been able to improve the manuscript's readability. We now provide more details in the analyses, show different figures with more informative content and the presentation of our contribution has improved considerably. We hope the editors and reviewers find the current version more appropriate for the journal.
%-------------------------------------

%%%%%%%%%%%%%%%%%%%
\noindent---------------------------------------------------------------------------------------
\section*{Associate Editor}

%-------------------------------------
\textit{Dear Authors,\\
I regret to inform you that I cannot recommend publication of your paper in the IEEE Transactions on Geoscience and
Remote Sensing journal. This rejection is based upon the enclosed comments of the reviewers.
In case you decide to resubmit the paper to this journal, I suggest that you take into account all the reviewers’ comments
to improve your manuscript.
Thank you for considering TGARS for the publication of your manuscript. We welcome the future submission of your
work to the journal.\\
Sincerely,}

\medskip

\textbf{R:} We would like to thank the associate editor for his consideration with our manuscript and for allowing us to resubmit our work. The manuscript was carefully reviewed and a thorough analysis was carried out to improve its content. The reviewers' comments have been addressed and below you can find an explanation to each one of them.
%-------------------------------------

\medskip

\medskip

\medskip

%\newpage

\vspace{0.25cm}

%%%%%%%%%%%%%%%%%%%
\noindent---------------------------------------------------------------------------------------
\section*{Reviewer \#1}

%-------------------------------------
\textit{This paper presents a wavelet decomposition based change detection method for POLSAR images. The method seems original but the paper is poorly written. The detailed comments are listed as follows.}

\medskip

\textbf{R:} We thank the reviewer for all his/her contributions, technical insights, and the time spent on reviewing our manuscript. It was rewritten and its content was carefully improved.
%-------------------------------------

\medskip


\medskip
%-------------------------------------
\begin{mybox}{Comment 1}
\textit{This paper lacks a thorough review of POLSAR image change detection methods. What are the challenges of
POLSAR image change detection? What are the advantages and defects of existing methods? What’s the superiority of
the proposed method over existing ones? What are the contributions of this paper to remote sensing community? There
are many works about change detection in POLSAR images. This paper only addresses the method without necessary
comparison.}

\medskip

\textbf{R:} The reviewer is totally right. Even though we wrote PolSAR in the title, the first manuscript did not address the characteristics of this type of image or mentioned related methods from the literature. The term PolSAR was removed from the manuscript and now we let it clear that our method is suitable for single channel images. We focus on SAR images in the application with real data. This is the only type of image mentioned in title now.

\medskip


\end{mybox}
%-------------------------------------

\vspace{0.3cm}

\medskip
%-------------------------------------
\begin{mybox}{Comment 2}
\textit{The authors claim that “it is sparse, very fast and scalable. Finally, it is easily adapted to be updatable”. However, the
change detection in POLSAR images has been significantly developed and there are many excellent methods. Many of
them are also fast and scalable. Without any necessary comparison, how to demonstrate that “it is very fast and scalable”? What’s the novelty of the proposed method?}

\medskip

\textbf{R:} These remarks about the proposed method were removed from the manuscript. Additionally, the current version does not refer to PolSAR images anymore.

\medskip

\end{mybox}
%-------------------------------------

\vspace{0.3cm}


\medskip
%-------------------------------------
\begin{mybox}{Comment 3}
\textit{What’s the relationship between this paper and [i]? [i] also uses wavelet transformation for change detection.
[i] Fonseca, R. , Ludwig, G. , Montoril, M. , Pinheiro, A. . (2020). Nonparametric methods for detecting change in Multitemporal SAR-PolSAR Satellite Data. arXiv: 2001.05764.}

\medskip

\textbf{R:} In the cited paper [i], the authors estimate probability density functions for each instant using the images' wavelet coefficients. Next, a time series of curves is estimated using such density functions, which are later used to compute a matrix of differences for pairs of instants. The method in the present manuscript is significantly simpler, basically using discrete wavelet transform for images and Pearson correlation, whereas [i] employs functional data analysis tools. 

\medskip


\end{mybox}
%-------------------------------------

\vspace{0.3cm}


\medskip
%-------------------------------------
\begin{mybox}{Comment 4}
\textit{Methodology describes the proposed method in detail, but what’s new in this paper? All the operators are existing or
intuitive.}

\medskip

\textbf{R:} The idea of using correlation screening to detect spatio/temporal changes has not been proposed in the literature yet. Works in feature screening basically employ existing and well known correlation measures, but the application of these techniques in the statistical learning literature is an active research topic regarding high-dimensional data. We show that this idea can offer useful results for satellite image data analysis as well.

\medskip


\end{mybox}
%-------------------------------------

\vspace{0.3cm}

\medskip
%-------------------------------------
\begin{mybox}{Comment 5}
\textit{The method is not clearly described. For example, what’s log-image? What’s the epsilon in Eq. (1). Eq. (1) seems
problematic. It denotes wavelet decomposition?}


\medskip
\textbf{R:} Log-image referred to the logarithm of a matrix representing an image, but such terminology is no longer used in the current manuscript. The epsilon mentioned referred to noise present in an observed image and it did not involve wavelet decomposition. We realized that such problems were undermining the presentation of our methodology. Therefore, we decided to rewrite this section and we hope it is more clear now.

\medskip

\end{mybox}
%-------------------------------------

\vspace{0.3cm}

\medskip
%-------------------------------------
\begin{mybox}{Comment 6}
\textit{Without comparison with recent change detection methods in experiments, how to demonstrate the advantages of the
proposed method? For example, if there is a method that is faster and performs better than existing ones, why I have to
use the proposed one?}


\medskip
\textbf{R:} We have not been able to find another method that address simultaneously spatial and temporal change detection for a full time series of satellite images. Most methods deal only with pairs of images. Therefore, we decided to compare our results with the standard approaches cited in the manuscript.

\medskip

\end{mybox}
%-------------------------------------

\vspace{0.3cm}

\medskip
%-------------------------------------
\begin{mybox}{Comment 7}
\textit{The authors claims that “Its performance is similar to a deep learning feature extraction method”. However, the deep
learning feature extraction method is proposed for image denoising and it is not specially designed for POLSAR images.
It is obviously unfair to compare the proposed method with it. Why not compare the proposed method with recent
POLSAR image change detection methods? The proposed one cannot outperform them?}


\medskip
\textbf{R:} The comparison with a deep learning method was removed, as well as any mention to PolSAR images.

\medskip

\end{mybox}
%-------------------------------------

\vspace{0.3cm}

\medskip
%-------------------------------------
\begin{mybox}{Comment 8}
\textit{For real data, there are only self comparisons without any comparison with existing excellent methods. Therefore, the
conclusions may be not correct.}


\medskip
\textbf{R:} The real data application was improved and now we offer much more details about the region analyzed. The analyses are once again compared with standard approaches for the type of analyses we make.

\medskip

\end{mybox}
%-------------------------------------


\medskip
\medskip
\medskip

%\newpage



\vspace{0.25cm}

%%%%%%%%%%%%%%%%%%%
\section*{Reviewer \#2}


%-------------------------------------
\begin{mybox}{General Comment}

\textit{Fonseca et al. present a wavelet-based approach to detect spatiotemporal change points on multi-temporal Polarimetric
SAR. The approach is potentially interesting and might be of use for a broader community. However, it suffered from
several shortcomings, which prevents me from recommending the publication of the manuscript in its current form. In the
following o elaborate my concerns, and I hope authors find them useful.}

\medskip

\textbf{R:} We thank the reviewer for all his/her contributions, technical insights and the time
spent on reviewing our manuscript. The manuscript was carefully improved and we hope the recent version is suitable to be considered for publication.
\end{mybox}
%-------------------------------------

\medskip
%-------------------------------------
\begin{mybox}{Comment 1}
\textit{The method section is unclear and partly incomplete. There are equations whose terms are not defined (e.g., Eq 1 and
that between 7 and 8 without a number). Furthermore, the procedure for Wavelet multiresolution analysis is not
described. Although, provided in the cited literature, it is crucial to carefully layout the procedure used to create the input
to the change detection algorithm. Also, statements like "J-th approximation level" are unclear. I suspect the author
meant to say "J-th decomposition level."}


\medskip
\textbf{R:} Presentation of the methodology was carefully revised and now all terms and methods are introduced before being referred to. We present a brief introduction of wavelet decomposition and feature screening that might be enough to understand further parts of the manuscript. Concerning the approximation level, it is indeed one of the decomposition levels, but we meant that only coefficients related to coarse details are used. This is explained with more details in the current version.
\end{mybox}
%-------------------------------------

\medskip

%-------------------------------------

\begin{mybox}{Comment 2}
\textit{Synthetic test is confusing. Figure 1 shows a sequence of 4 synthetic images with changes appearing consecutively.
The expectation from the presented method and promises made in the abstract and introduction is that 4 corresponding
change maps can be obtained characterizing the changes made in the synthetic datasets. However, Figure 2 shows a
total change and some other maps, which is unclear how they were obtained or what is the relevance to the presented
method and promises of the paper.}

\medskip

\textbf{R:} The four images are now presented separately in Figure 2 of the recent version. The total changes are showed in Figure 3 of this version. The idea is that, if only the first and last images were used (images 1 and 4, respectively), changes in instants between these times would be missed. The proposed method is suited to detect this type of change only observed when considering the full time series of images. Explanations were improved in the section about the simulated images and we added more details on how they were generated.
\end{mybox}
%-------------------------------------


%-------------------------------------

\vspace{0.3cm}


\medskip
%-------------------------------------

\begin{mybox}{Comment 3}
\textit{The real data set is not introduced carefully, and it is not clear what datasets from what sensor at what resolution and
quality are used. How the real data is proceeded and prepared for this analysis, etc. In Figures 7-9 most of the panels
(e.g., c-j) show many overlapping curves that color-coded without providing a colorbar. These figures carry very little
useful information and I encourage authors to rethink ways to present their results. I also suggest authors to provide
examples of images and show the wavelet decomposition results. Also, show the average approximate coefficient
matrix. Last but not least, quantify a change map and provide a validation test such as an independent optical image
taken from the same area...}

\medskip

\textbf{R:} Details about the images were included in the text:
the acquisition satellite/sensor; spatial resolution; and the image size.
% type of satellite image; type of sensor; resolution; \textcolor{red}{and description of pre-processing steps used before applying the proposed method}. 
 We also removed similar figures that were not adding new information to the analysis. The figures with overlapping curves are no longer used, and different details about the region studied are presented now, including change maps in Figures 6(c) and 10. In particular, Figure 6(c) is a validation/reference change map that is used to compute accuracy measure of the methods considered. These measures are given in Table 1 of the revised manuscript.% recent manuscript version.
\end{mybox}
%-------------------------------------


%\newpage
\medskip
\medskip
\medskip


%-------------------------------------
%%%%%%%%%%%%%%%%%%%
\section*{Reviewer \#3}


%-------------------------------------
\begin{mybox}{General Comment}

\textit{The manuscript presents a methodology to detect changes in the wavelet domain using multitemporal SAR
data. Although all the mathematical formalism looks sound and the results on simulated data are promising,
the manuscript is still lacking of some important parts. Most importantly, a proper analysis of results using
real data should be done. Real data are indeed used, but their analysis has little to do with validating change
detection and it is more about showing how the new observables look like. More information should also be
added to the simulations. After the addition of these information and a proper validation the paper should be
publishable in TGRS.}
\medskip

\textbf{R:} We thank the reviewer for all his/her contributions, technical insights and the time
spent on reviewing our manuscript. The analysis of real data was improved and now we offer more details about the images used and present a validation change image to measure the accuracy of the proposed method. We also provide now a better description of how the simulated images were generated and analyzed.
\end{mybox}
%-------------------------------------

\medskip
%-------------------------------------
\begin{mybox}{Comment 1}
\textit{Data used: In the entire paper there is no mention of the type of data used. Are this Sentinel-1 GRD images? Also please consider acknowledging ESA for the use of data.}


\medskip
\textbf{R:} The reviewer is correct, the Sentinel-1 GRD images were employed in this research. The revised text includes a description of this sensor/dataset. Acknowledgments to ESA were also included at the end of the manuscript. Thank you.
\end{mybox}
%-------------------------------------

\medskip
%-------------------------------------
\begin{mybox}{Comment 2}
\textit{Simulations: Some more details should be provided to describe how the simulations were performed. I assume these are Monte Carlo simulations where an Additive Gaussian Noise is added to the image (representing
the log intensity). This is not ideal, since it opens the question on the noise level you simulate (i.e. the energy
of the noise). Why did you selected that energy for the noise? Ideally, you would simulate the complex pixel
as a Circular Complex Gaussian random variable, then take the intensity (maybe use multi-look) and then the
log. Please justify the value of noise energy you selected or please repeat the simulations starting from complex
pixels.}


\medskip
\textbf{R:} The simulated images were indeed generated with additive Gaussian noise. We included more details about how the simulated images were generated and what level of noise was used. The signal to noise to ratio considered was chosen in order to not let the true changes too apparent in the observed images (especially image 4), as can be seen in Figure 2. Such level of noise is also more challenging to analyze than what is observed in the application with real data. Such explanations are presented in the manuscript now.   
\end{mybox}
%-------------------------------------

\medskip
%-------------------------------------
\begin{mybox}{Comment 3}
\textit{Figure 2: Is (a) a validation mask where each pixel that changes at any time is flagged as 1? Please add
more details on the content of subplot (a).}


\medskip
\textbf{R:} Yes, ones (white) correspond to locations were change happened at any instant, whereas zeros (black) correspond to places where no change takes place.  This information is given now in the recent version and the figure now is on Figure 3a. We also present more details about the content of the remaining images as well.
\end{mybox}
%-------------------------------------

\medskip
%-------------------------------------
\begin{mybox}{Comment 4}
\textit{Timeseries vs pairwise: At the moment, it is not clear what is the advantage of using the full time series
compared to apply a pairwise approach using first and last image. In the simulation, is this a better choice in
detecting changes that applying a single pairwise change detector between image 1 and image 4? So not do the
accumulation, but just use first and last change detection for comparison.}


\medskip
\textbf{R:} The simulated images illustrate the point in considering the full time series: if only images 1 and 4 were considered in a pairwise analysis, the changes that happened between these instants would be missing. The proposed method is able to detect this type of change, as can be seen in Figure 3b. Additionally, we also look for instants when a relevant change might have happened. The method proposed can be used to detect this type of change as well through analyzing the the measures $\mathbf{d}^{(1)},\ldots,\mathbf{d}^{(n)}$.
\end{mybox}
%-------------------------------------

\medskip
%-------------------------------------
\begin{mybox}{Comment 5}
\textit{Real data info and preprocessing: There is absolutely no information about what satellite data are used. I
assume this are Sentinel-1 GRD, but please provide info on that. Also there is no info on what pre-processing
you use. Did you do calibration, terrain correction, noise removal, geocoding?}


\medskip
\textbf{R:} As mentioned (in comment \#1), we added more details about the adopted images in the experiment with actual remote sensing data.

% used in the real data application. They are indeed Sentinel-1 images.

%\textcolor{green}{Não houve pre-processamento, mas é interessante incluir isso no texto. Eu farei isso.}
\end{mybox}
%-------------------------------------

\medskip
%-------------------------------------
\begin{mybox}{Comment 6}
\textit{Real data validation: The test with real data is very poor and should be improved. For instance, there is
no information on what changed in the scene. Also there is no detection mask obtained from the analysis and
no quantitative results (e.g. a ROC curve). These informations must be provided in order to make this paper
publishable.}


\medskip
\textbf{R:} A validation image with change locations is used in the recent version and can be seen in Figure 6(c). This image is used to compute accuracy measures of the proposed method and standard approaches. A change map is also provided using two different types of thresholding algorithms (Otsu and Kittler-Illingworth).
\end{mybox}
%-------------------------------------

\medskip
%-------------------------------------
\begin{mybox}{Comment 7}
\textit{Figure 4 to 6: These are temporal trends, but there is no information on what the reader is supposed to see
in these trends. If you take the backscattering (or any other measure, e.g. kurtosis) it will always show a trend...
why are the trends coming from your method better than any other trend one could get?\\
Each plot I assume is referred to a region. What is the region you are looking at (forest, river, agriculture)?
How big is the region? Are we supposed to have changes in these regions? If so, which changes and when they
happen?}


\medskip
\textbf{R:} These figures are not presented in the recent version, and only an overall change measure is presented in Figure 7. An increasing trend in this plot means that the corresponding images are changing with respect to the mean image. A decreasing trend indicates that images are becoming more similar to the mean image.
\end{mybox}
%-------------------------------------

\medskip
%-------------------------------------
\begin{mybox}{Comment 8}
\textit{Figure 7 to 9: What do all these colour lines in the plots refer too? Are different different regions?}


\medskip
\textbf{R:} Yes, each color corresponds to a different location (pixel). The goal was to show that locations flagged as change point have squared deviations that behave similarly as the overall change measure. Since these figures were somewhat confusing, we decided not to show them in the recent version.
\end{mybox}
%-------------------------------------

\medskip
%-------------------------------------
\begin{mybox}{Comment 9}
\textit{Figure 10 and 11: Do you have information on the area that changed? If you want to use R (d) for detection,
and low correlation are flagged as changing areas, doe s this mean that the river (permanent water body) is the
most changing part?}

\medskip
\textbf{R:} We present now a change map that is used to evaluate the methods. When using $\mathbf{R}$, high correlations are the ones flagged as corresponding to changing areas. This is because the correlation is computed between a local change measure and an overall change measure. Therefore, the river is not detected as changing region by the proposed method.
\end{mybox}
%-------------------------------------

\medskip
%-------------------------------------
\begin{mybox}{Comment 10}
\textit{P.2, c.1, l.31: Please change "practive" into "practice".}


\medskip
\textbf{R:} The word was corrected in the recent version.
\end{mybox}
%-------------------------------------




\vspace{0.3cm}


\bibliographystyle{apalike}
\bibliography{../JARS_DA_review/refsartigo}


\end{document}



