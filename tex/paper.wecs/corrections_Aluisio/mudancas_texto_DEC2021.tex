PAGE 1 COLUMN 1 

* LINE 24 "to images available at larger and larger from time to time" ==> "to large images available over long periods".

PAGE 1 COLUMN 2

* LINE 7 "not affected" ==> "unaffected"

* LINE 18 "is" XXXXX

* LINE 20 "In more details," XXXXX

* LINES 21-22 "not only ... problems" XXXXXX

* LINES 33-34 "In the light ... concepts" ==> "Given the aforementioned discussion and motivation"

* LINE -2 "A basic" ==> "basic"

PAGE 2 COLUMN 1

PARAGRAPH 3 

* LINE 8 'needed" ==> "desirable".

PARAGRAPH 4  

* LINE 1 "In details," XXXXX

* LINE 2 "for ultra-high" ==> "for high and ultra-high"

* LINE 7 ", where " XXXXX

* LINE 9 "is that" ==> "is such that"

* LINE 9 "what" ==> "which"

* LINE 10 "unfeasible, but only" ==> "unfeasible. Moreover, only" 

* LINE -3 "could" ==> "should"

PAGE 2 COLUMN 2

PARAGRAPH 2

* LINE 9 "along" ==> "over" 

PARAGRAPH 3

* LINE 1 "After" ==> "The"

* LINE 2 ", one of its byproducts is" ==> "results in"

* LINE 3 ", of so" ==> "of the so"

PARAGRAPH 4

* LINE 1 "can be" ==> "may be"

* LINE 2 "analysis" ==> "analyses"

* LINES 2-3 "different types of wavelet transform and basis" ==> "different wavelet bases"

* LINE 3 "as well the use of" ==> "as well as the"

* LINE 4 "for detail" ==> "of detail"

PAGE 3 COLUMN 1

PARAGRAPH 2

* LINE 1 "The most right-hand term in " ==> "The last term on the right-hand side of "

* LINE 3 "regarding" ==> "and"

* LINE 5 "the time" ==> "time"

* LINE 8 "the spatial" ==> "spatial"

* LINES 9-10 "the Pearson correlation coefficient rises as an convenient alternative" ==> 
"the Pearson correlation coefficient is an interesting alternative".

* LINE -0 ", as explained earlier" ==> "[26]"

* FORMULAS (2) e (3) Faltam pontos finais.

PARAGRAPH 6

* LINES 1-2  "Consequen ... R ... assigned" ==> "Let R=[R_{kl}], where R_{kl} is assigned"

PARAGRAPH 7

* LINE 1 "Let us define" ==> "Define"

* LINE -2 "that for " ==> "that, for

* ACRESCENTAR AQUI A IDEIA DE REGRESSAO ... ******Est� no final do arquivo

PAGE 3 COLUMN 2

PARAGRAPH 2

* LINE 2 "distinct studies" ==> "studies"

PARAGRAPH 3

* LINE 2 "identify" ==> "identifying"

PARAGRAPH 3

* LINES -1-(-2) "Posterior ... out" XXXXX

PARAGRAPH 4

* LINE 1 "Lastly, the" ==> "The"

 PAGE 4 COLUMN 1
 
 PARAGRAPH 1
 
* LINES 1-2 "All the changes made that occur among subsequent" ==> "All changes between subsequent"
 
* LINE 3 "the "ones")" ==> " "ones")"
 
* LINE 4 "along ...instants" XXXXXXX
 
 PARAGRAPH 2 
 
* LINE 1 "images,"  ==> "images"
 
  PARAGRAPH 2 
 
* LINE 3 "The curves allow concluding that" ==> "We conclude that"
 
 PAGE 4 COLUMN 2
 
 PARAGRAPH 2
 
* LINE -1 "to the competitors" ==> "both competitors"
 
 PARAGRAPH 3
 
* LINE 1 "This sections demonstrates the WECS" ==> "This sections compares the performance of WECS"

* LINE 2 "methods" XXXXXXX

* LINE 3 "In detail, this application considers" ==> "We consider here 

PAGE 5 COLUMN 1
 
 PARAGRAPH 2
 
* LINES 3-4 "Figures 9(a)-9(b)" ==> "Figures 6(a)-6(b)"
 
 PARAGRAPH 3
 
* LINE 2 "it was" ==> "it is"
 
 PARAGRAPH 4
 
* LINE -2 "Figure 7" ==> "Figure 8"
 
 PAGE 5 COLUMN 2
 
 PARAGRAPH 2
 
* LINES 1-3 "After submitting the image series to the methods WECS, ECS, and TAAD, are obtained ... domain" ==> "Through the application WECS, ECS and TAAD to the image series, we obtain matrices R, \tilde{R} and A. They represent the spatially-localized change measures given by WECS, ECS and TAAD, respectively".

PAGE 6 COLUMN 1
 
 PARAGRAPH 2
 
* LINE 4 "kappa coefficient with basis" ==> "kappa coefficient based"

PARAGRAPH 3
 
* LINES 1-2 "the WECS method" ==> "WECS"
 
 PAGE 6 COLUMN 2
 
 PARAGRAPH 1
 
* LINE 5 "the WECS method" ==> "WECS"

 PARAGRAPH 2
 
* LINE 4 "than ECS and TAAD" ==> "than either ECS or TAAD".
 
* LINES 4-5 "the TAAD method" ==> "TAAD"
 
 PARAGRAPH 3
 
* LINE 1 "the WECS method" ==> "WECS"

PAGE 7 COLUMN 2
 
 PARAGRAPH 3 
 
* LINE 3-6 "studying energy correlation screening method with for distinct smoothing techniques; deducing change detection rates theoretically for a statistical models" ==> "extending energy correlation screening for distinct smoothing techniques; deducing sharp theoretical change detection rates for appropriate statistical models"
 
************************************ IDEIA DE REGRESSAO *****************

Na p�gina 3 coluna 1 o antepen�ltimo par�grafo se l� " Define a mapping of {\it relevant} indices ... as ${\mathcal M}_* ...$ where ... in the images".

Aqui, acrescenta-se


"If we apply to Equation (2) the index dicothomy defined by ${\mathcal M}_*$, we have
\begin{equation}
d(m)=\sum_{(k,l)\in\mathcal M_*}\beta_{k,l}D_{k,l}^{(m)}+\varepilon(m), \label{Regressao}
\end{equation}
where $\beta_{k,l}$ are non-null regression coefficients for $(k,l)\in\mathcal M_*$, and $\varepilon(m)$ are stochastic error terms. 

The error terms allow for both the apportionment of spurious correlation for indices $notin \mathcal M_*$ as well as for the energies not represented by the wavelet smoothing.

It is a well-known property of discrete wavelet transforms that it statistically decorrelates the original data [19,]-[20]. For instance, this motivates the use of WECS instead of ECS or TAAD, since wavelets will result in sparser representations for $\mathcal M_*$ against either method. Moreover, the sure screening theoretical results for independent data motivates our conjecture that the regression set-up given by Equation (\ref{Regressao}) should have a good performance. A rigorous proof for dependent data sets such as multitemporal series of satellite images is beyond the scope of this manuscript, but the numerical results provide us with some solid evidences."



 
 
